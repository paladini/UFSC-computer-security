%%% Template originaly created by Karol Kozioł (mail@karol-koziol.net) and modified for ShareLaTeX use

\documentclass[a4paper,11pt]{article}

\usepackage[T1]{fontenc}
\usepackage[utf8]{inputenc}
\usepackage{graphicx}
\usepackage{xcolor}

\renewcommand\familydefault{\sfdefault}
\usepackage{tgheros}
\usepackage[defaultmono]{droidmono}

\usepackage{amsmath,amssymb,amsthm,textcomp}
\usepackage{enumerate}
\usepackage{multicol}
\usepackage{tikz}
\usepackage{geometry}
\geometry{total={210mm,297mm},
left=25mm,right=25mm,%
bindingoffset=0mm, top=20mm,bottom=20mm}


\linespread{1.3}

\newcommand{\linia}{\rule{\linewidth}{0.5pt}}

% custom theorems if needed
\newtheoremstyle{mytheor}
    {1ex}{1ex}{\normalfont}{0pt}{\scshape}{.}{1ex}
    {{\thmname{#1 }}{\thmnumber{#2}}{\thmnote{ (#3)}}}

\theoremstyle{mytheor}
\newtheorem{defi}{Definition}

% my own titles
\makeatletter
\renewcommand{\maketitle}{
\begin{center}
\vspace{2ex}
{\huge \textsc{\@title}}
\vspace{1ex}
\\
\linia\\
\@author \hfill \@date
\vspace{4ex}
\end{center}
}
\makeatother
%%%

% custom footers and headers
\usepackage{fancyhdr}
\pagestyle{fancy}
\lhead{}
\chead{}
\rhead{}
\lfoot{Trabalho \textnumero{} 1}
\cfoot{}
\rfoot{Página \thepage}
\renewcommand{\headrulewidth}{0pt}
\renewcommand{\footrulewidth}{0pt}
%

% code listing settings
\usepackage{listings}
\lstset{
    language=Python,
    basicstyle=\ttfamily\small,
    aboveskip={1.0\baselineskip},
    belowskip={1.0\baselineskip},
    columns=fixed,
    breaklines=true,
    tabsize=4,
    prebreak=\raisebox{0ex}[0ex][0ex]{\ensuremath{\hookleftarrow}},
    frame=lines,
    showtabs=false,
    showspaces=false,
    showstringspaces=false,
    keywordstyle=\color[rgb]{0.627,0.126,0.941},
    commentstyle=\color[rgb]{0.133,0.545,0.133},
    stringstyle=\color[rgb]{01,0,0},
    numbers=left,
    numberstyle=\small,
    stepnumber=1,
    numbersep=10pt,
    captionpos=t,
    inputencoding=utf8,
    extendedchars=true,
    escapeinside={\%*}{*)}
}

\usepackage{seqsplit}

%%%----------%%%----------%%%----------%%%----------%%%

\begin{document}

\title{Trabalho Individual sobre OpenSSL}

\author{Fernando Paladini, Segurança em Computação (INE5429)}

\date{01/10/2016}

\maketitle

\section*{1) Comando asn1parse}

O comando ASN1PARSE da ferramenta OpenSSL tem como finalidade analisar (\textit{parsear}, \textit{varrer}) e extrair dados de estruturas descritas em ASN.1 (\textit{Abstract Syntax Notation 1}). Dentre as opções disponíveis para esse comando estão:\newline

\noindent \textbf{-inform <formato>}: o formato do arquivo de entrada, que pode ter como valor DER (binário) ou PEM (arquivo codificado em base64). O valor padrão é PEM.
\newline\newline
\noindent \textbf{-in <arquivo>}: o único argumento é o arquivo de entrada (<arquivo>).
\newline\newline
\noindent \textbf{-out <arquivo>}: o arquivo de saída onde os dados codificados em DER serão armazenados. Se esta opção não estiver presente, nenhuma saída não será gerada.
\newline\newline
\textbf{-noout}: não gera uma saída.
\newline\newline
\textbf{-offset <numero>}: o \textit{offset} dado antes de começar a varrer (\textit{parsear}) o arquivo. Por padrão é o começo do arquivo.
\newline\newline
\textbf{-length <numero>}: número de bytes para varrer (\textit{parsear}). Por padrão é até o final do arquivo.
\newline\newline
\textbf{-i}: indenta a saída de acordo com a profundidade da estrutura dada como arquivo de entrada.
\newline\newline
\textbf{-dump}: fazer o dump de dados em formato hexadecimal.
\newline\newline
\textbf{-strparse <offset>}: varre (\textit{parse}) os octetos do conteúdo do objeto ASN.1 a partir do <offset> fornecido. Essa opção pode ser usada múltiplas vezes.
\newline\newline
\textbf{-genstr <string>, -genconf <arquivo>}: gera dados codificados em <string>, <arquivo> ou em ambos utilizando o formato ASN.1.

\noindent Exemplo de comando:
\begin{lstlisting}
# 
$ openssl asn1parse -inform PEM -in directbooking_co.crt -i
    0:d=0  hl=4 l=1401 cons: SEQUENCE          
    4:d=1  hl=4 l=1121 cons:  SEQUENCE          
    8:d=2  hl=2 l=   3 cons:   cont [ 0 ]        
   10:d=3  hl=2 l=   1 prim:    INTEGER           :02
   13:d=2  hl=2 l=  17 prim:   INTEGER           :CAFBF1C9B17442D25C179688F808A2EC
   32:d=2  hl=2 l=  13 cons:   SEQUENCE          
   34:d=3  hl=2 l=   9 prim:    OBJECT            :sha256WithRSAEncryption
   45:d=3  hl=2 l=   0 prim:    NULL              
   47:d=2  hl=3 l= 144 cons:   SEQUENCE          
   50:d=3  hl=2 l=  11 cons:    SET               
   52:d=4  hl=2 l=   9 cons:     SEQUENCE          
   54:d=5  hl=2 l=   3 prim:      OBJECT            :countryName
   59:d=5  hl=2 l=   2 prim:      PRINTABLESTRING   :GB
   63:d=3  hl=2 l=  27 cons:    SET               
   65:d=4  hl=2 l=  25 cons:     SEQUENCE          
   67:d=5  hl=2 l=   3 prim:      OBJECT            :stateOrProvinceName
   72:d=5  hl=2 l=  18 prim:      PRINTABLESTRING   :Greater Manchester
   92:d=3  hl=2 l=  16 cons:    SET               
   94:d=4  hl=2 l=  14 cons:     SEQUENCE          
   96:d=5  hl=2 l=   3 prim:      OBJECT            :localityName
  101:d=5  hl=2 l=   7 prim:      PRINTABLESTRING   :Salford
  110:d=3  hl=2 l=  26 cons:    SET               
  112:d=4  hl=2 l=  24 cons:     SEQUENCE          
  114:d=5  hl=2 l=   3 prim:      OBJECT            :organizationName
  119:d=5  hl=2 l=  17 prim:      PRINTABLESTRING   :COMODO CA Limited
  138:d=3  hl=2 l=  54 cons:    SET               
  140:d=4  hl=2 l=  52 cons:     SEQUENCE          
  142:d=5  hl=2 l=   3 prim:      OBJECT            :commonName
  147:d=5  hl=2 l=  45 prim:      PRINTABLESTRING   :COMODO RSA Domain Validation Secure Server CA
  194:d=2  hl=2 l=  30 cons:   SEQUENCE          
  196:d=3  hl=2 l=  13 prim:    UTCTIME           :160204000000Z
  211:d=3  hl=2 l=  13 prim:    UTCTIME           :170203235959Z
  226:d=2  hl=2 l=  97 cons:   SEQUENCE          
  228:d=3  hl=2 l=  33 cons:    SET               
  230:d=4  hl=2 l=  31 cons:     SEQUENCE          
  232:d=5  hl=2 l=   3 prim:      OBJECT            :organizationalUnitName
  237:d=5  hl=2 l=  24 prim:      PRINTABLESTRING   :Domain Control Validated
  263:d=3  hl=2 l=  33 cons:    SET               
  265:d=4  hl=2 l=  31 cons:     SEQUENCE          
  267:d=5  hl=2 l=   3 prim:      OBJECT            :organizationalUnitName
  272:d=5  hl=2 l=  24 prim:      PRINTABLESTRING   :PositiveSSL Multi-Domain
  298:d=3  hl=2 l=  25 cons:    SET               
  300:d=4  hl=2 l=  23 cons:     SEQUENCE          
  302:d=5  hl=2 l=   3 prim:      OBJECT            :commonName
  307:d=5  hl=2 l=  16 prim:      PRINTABLESTRING   :directbooking.co
  325:d=2  hl=4 l= 290 cons:   SEQUENCE          
  329:d=3  hl=2 l=  13 cons:    SEQUENCE          
  331:d=4  hl=2 l=   9 prim:     OBJECT            :rsaEncryption
  342:d=4  hl=2 l=   0 prim:     NULL              
  344:d=3  hl=4 l= 271 prim:    BIT STRING        
  619:d=2  hl=4 l= 506 cons:   cont [ 3 ]        
  623:d=3  hl=4 l= 502 cons:    SEQUENCE          
  627:d=4  hl=2 l=  31 cons:     SEQUENCE          
  629:d=5  hl=2 l=   3 prim:      OBJECT            :X509v3 Authority Key Identifier
  634:d=5  hl=2 l=  24 prim:      OCTET STRING      [HEX DUMP]:3016801490AF6A3A945A0BD890EA125673DF43B43A28DAE7
  660:d=4  hl=2 l=  29 cons:     SEQUENCE          
  662:d=5  hl=2 l=   3 prim:      OBJECT            :X509v3 Subject Key Identifier
  667:d=5  hl=2 l=  22 prim:      OCTET STRING      [HEX DUMP]:04140C56EA09820200C68F0CFE499D592F02DE5055F1
  691:d=4  hl=2 l=  14 cons:     SEQUENCE          
  693:d=5  hl=2 l=   3 prim:      OBJECT            :X509v3 Key Usage
  698:d=5  hl=2 l=   1 prim:      BOOLEAN           :255
  701:d=5  hl=2 l=   4 prim:      OCTET STRING      [HEX DUMP]:030205A0
  707:d=4  hl=2 l=  12 cons:     SEQUENCE          
  709:d=5  hl=2 l=   3 prim:      OBJECT            :X509v3 Basic Constraints
  714:d=5  hl=2 l=   1 prim:      BOOLEAN           :255
  717:d=5  hl=2 l=   2 prim:      OCTET STRING      [HEX DUMP]:3000
  721:d=4  hl=2 l=  29 cons:     SEQUENCE          
  723:d=5  hl=2 l=   3 prim:      OBJECT            :X509v3 Extended Key Usage
  728:d=5  hl=2 l=  22 prim:      OCTET STRING      [HEX DUMP]:301406082B0601050507030106082B06010505070302
  752:d=4  hl=2 l=  79 cons:     SEQUENCE          
  754:d=5  hl=2 l=   3 prim:      OBJECT            :X509v3 Certificate Policies
  759:d=5  hl=2 l=  72 prim:      OCTET STRING      [HEX DUMP]:3046303A060B2B06010401B23101020207302B302906082B06010505070201161D68747470733A2F2F7365637572652E636F6D6F646F2E636F6D2F4350533008060667810C010201
  833:d=4  hl=2 l=  84 cons:     SEQUENCE          
  835:d=5  hl=2 l=   3 prim:      OBJECT            :X509v3 CRL Distribution Points
  840:d=5  hl=2 l=  77 prim:      OCTET STRING      [HEX DUMP]:304B3049A047A0458643687474703A2F2F63726C2E636F6D6F646F63612E636F6D2F434F4D4F444F525341446F6D61696E56616C69646174696F6E53656375726553657276657243412E63726C
  919:d=4  hl=3 l= 133 cons:     SEQUENCE          
  922:d=5  hl=2 l=   8 prim:      OBJECT            :Authority Information Access
  932:d=5  hl=2 l= 121 prim:      OCTET STRING      [HEX DUMP]:3077304F06082B060105050730028643687474703A2F2F6372742E636F6D6F646F63612E636F6D2F434F4D4F444F525341446F6D61696E56616C69646174696F6E53656375726553657276657243412E637274302406082B060105050730018618687474703A2F2F6F6373702E636F6D6F646F63612E636F6D
 1055:d=4  hl=2 l=  72 cons:     SEQUENCE          
 1057:d=5  hl=2 l=   3 prim:      OBJECT            :X509v3 Subject Alternative Name
 1062:d=5  hl=2 l=  65 prim:      OCTET STRING      [HEX DUMP]:303F8210646972656374626F6F6B696E672E636F82146170702E646972656374626F6F6B696E672E636F8215617070322E646972656374626F6F6B696E672E636F
 1129:d=1  hl=2 l=  13 cons:  SEQUENCE          
 1131:d=2  hl=2 l=   9 prim:   OBJECT            :sha256WithRSAEncryption
 1142:d=2  hl=2 l=   0 prim:   NULL              
 1144:d=1  hl=4 l= 257 prim:  BIT STRING      
\end{lstlisting}

\section*{2) Comando ciphers}

O comando CIPHERS da ferramenta OpenSSL tem como finalidade converter uma lista textual de cifradores OpenSSL para uma lista ordenada de preferências de cifradores OpenSSL.

\noindent \textbf{-v}: opção com \textit{verbosidade}, ou seja, imprimindo na tela muito mais informações do que está acontecendo, incluindo descrição da versão do protocolo, autenticação, encriptação, etc. 
\newline\newline
\noindent \textbf{-ssl3}: lista os cifradores compatíves com o SSLv3.
\newline\newline
\noindent \textbf{-tls1}: lista os cifradores compatíves com TLSv1, TLSv1.1 ou TLSv1.2.
\newline\newline
\noindent \textbf{-ssl2}: lista apenas os cifradores compatíves com SSLv2.
\newline\newline
\noindent \textbf{<cipherlist>}: uma lista de cifradores que será convertida para para uma lista de preferências de cifradores. Se nenhum argumento for passado, então a lista de cifradores padrão será utilizada. Essa lista consiste de strings separadas por dois pontos (\textit{:}), virgulas ou espaço.

\noindent Exemplo de comando:
\begin{lstlisting}
$ openssl ciphers -v ECDH+aRSA+HIGH
ECDHE-RSA-AES256-GCM-SHA384 TLSv1.2 Kx=ECDH     Au=RSA  Enc=AESGCM(256) Mac=AEAD
ECDHE-RSA-AES256-SHA384 TLSv1.2 Kx=ECDH     Au=RSA  Enc=AES(256)  Mac=SHA384
ECDHE-RSA-AES256-SHA    SSLv3 Kx=ECDH     Au=RSA  Enc=AES(256)  Mac=SHA1
ECDHE-RSA-AES128-GCM-SHA256 TLSv1.2 Kx=ECDH     Au=RSA  Enc=AESGCM(128) Mac=AEAD
ECDHE-RSA-AES128-SHA256 TLSv1.2 Kx=ECDH     Au=RSA  Enc=AES(128)  Mac=SHA256
ECDHE-RSA-AES128-SHA    SSLv3 Kx=ECDH     Au=RSA  Enc=AES(128)  Mac=SHA1
ECDHE-RSA-DES-CBC3-SHA  SSLv3 Kx=ECDH     Au=RSA  Enc=3DES(168) Mac=SHA1
\end{lstlisting}


\section*{3) Comando dgst}

O comando DGST da ferramenta OpenSSL tem como objetivo gerar uma mensagem sintetizada de um arquivo de origem, além de também poder gerar e verificar assinaturas digitais.
\newline\newline
\noindent \textbf{-c}:  printa na saída padrão grupos de dois digitos separados por dois pontos. Essa opção apenas é relevante se um formato de saída em hexadecimal é usado.
\newline\newline
\noindent \textbf{-d}:  printa na saída padrão informações de debug do comando. 
\newline\newline
\noindent \textbf{-hex}: a saída será gerada como um dump em hexadecimal. Esse é o caso normal para uma síntese "normal", diferente de uma assinatura digital.
\newline\newline
\noindent \textbf{-binary}: gera uma saída de síntese ou assinatura digital em binário.
\newline\newline
\noindent \textbf{-r}: gera uma saída síntese no formato de "coreutils", que é utilizado em programas como sha1sum.
\newline\newline
\noindent \textbf{-non-fips-allow}: permite o uso de síntese não FIPS quando o mode FIPS está ativado. Não tem efeito nenhum quando o modo FIPS não está ativado. 
\newline\newline
\noindent \textbf{-out <arquivo>}: arquivo onde a saída será escrita. Por padrão utiliza a saída padrão. 
\newline\newline
\noindent \textbf{-sign <arquivo>}: assina digital o arquivo utilizando a chave privada fornecida no argumento <arquivo>.
\newline\newline
\noindent \textbf{-keyform <arg>}: especifica o formato da chave utiliza para assinar a síntese. Os formatos DER, PEM, P12 e ENGINE são suportados por essa opção.
\newline\newline
\noindent \textbf{-engine <id>}: use o motor fornecido no argumento <id> para operações (incluindo armazenamento da private key). Esse motor não é usado como código para algoritmos de síntese, a menos que ele também esteja especificado no arquivo de configuração.
\newline\newline
\noindent \textbf{-sigopt nm:v}: passa opções para o algoritmo de assinatura usando "assinar" ou "verificar". Os nomes e valores para essas opções são especificadas para cada algoritmo.
\newline\newline
\noindent \textbf{-passin arg}: A fonte da senha da chave privada.
\newline\newline
\noindent \textbf{-verify <arquivo>}: verificar a assinatura usando a chave pública contida em <arquivo>. A saída pode ser "Falha na verificação" ou ainda "Verificação OK".
\newline\newline
\noindent \textbf{-prverify <arquivo>}: verifica a assinatura usando a chave privada contida em <arquivo>.
\newline\newline
\noindent \textbf{-signature <arquivo>}: a assinatura a ser verificada.
\newline\newline
\noindent \textbf{-hmac <chave>}: cria um MAC hasheado usando o argumento <chave>.
\newline\newline
\noindent \textbf{-mac <alg>}: cria um (Message Authentication Code). O algoritmo de MAC mais popular é o HMAC, mas existem outros algoritmos de MAC que não são baseados em hash, como por exemplo o algoritmo gost-mac. 
\newline\newline
\noindent \textbf{-macopt <nm:v>}: passa argumentos para o algoritmo MAC especificado pela opção "-mac".
\newline\newline
\noindent \textbf{-rand <file(s)>}: um arquivo ou vários arquivos contendo dados aleatórios.
\newline\newline
\noindent \textbf{-non-fips-allow}: ativa o uso de algoritmos não FIPS, tais como MD5.
\newline\newline
\noindent \textbf{-fips-fingerprint}: computa o HMAC usando a chave especifica para certas operações OpenSSL-FIPS.
\newline\newline
\noindent \textbf{file...}: arquivo ou arquivos para serem usados de entrada. Se nenhum arquivo é especificado, então a entrada padrão é utilizada.
\newline

\noindent Exemplo de comando:
\begin{lstlisting}
# Gera um hash MD5 para o arquivo cachorro.gif
$ openssl dgst -md5 cachorro.gif
MD5(cachorro.gif)= a303ed7ce439738c2ce9f0791d9799c1

# Gera um hash SHA1 para o arquivo cachorro.gif
$ openssl dgst -sha1 cachorro.gif
SHA1(cachorro.gif)= e5cba219bad315b7d0d6e0912a2d423ee9801611
\end{lstlisting}

\section*{4) Comando enc}

O comando ENC da ferramenta OpenSSL tem como objetivo fornecer meios para encriptar ou descriptografar vários blocos ou cifradores stream baseado em senhas ou explicitamente fornecidos. Encriptação e descriptografia também podem ser feitas em Base64.

\noindent \textbf{-in <arquivo>}: o arquivo de entrada. Por padrão o valor é a entrada padrão.
\newline\newline
\noindent \textbf{-out <arquivo>}: o arquivo de saída. Por padrão o valor é saída padrão.
\newline\newline
\noindent \textbf{-pass <arg>}: a fonte da senha.
\newline\newline
\noindent \textbf{-salt}: usa um salt nas rotinas de derivação de chaves. Este é o valor default.
\newline\newline
\noindent \textbf{-nosalt}: não usa salt nas rotinas de derivação de chaves. Está opção NÃO deve ser usado exceto para testes ou compatibilidade com versões antigas do OpenSSL e SSLeay.
\newline\newline
\noindent \textbf{-e}: encripta os dados de entrada. Esse é o padrão.
\newline\newline
\noindent \textbf{-d}: descriptografa os dados de entrada.
\newline\newline
\noindent \textbf{-a}: processa os dados em base64. Isso significa que se os dados estão codificados em base64.
\newline\newline
\noindent \textbf{-base64}: sinônimo para "-a".
\newline\newline
\noindent \textbf{-A}:  se a opção "-a" estiver presente, então a base64 é processada em uma linha.
\newline\newline
\noindent \textbf{-k <senha>}: a senha é derivada da chave. Esta opção é útil para versões antigas de OpenSSL. 
\newline\newline
\noindent \textbf{-kfile <arquivo>}: lê a primeira linha do arquivo fornecido em <arquivo> e o utiliza de senha para derivar a chave. Esta opção é útil para compatibilidade com versões antigas do OpenSSL.
\newline\newline
\noindent \textbf{-nosalt}: não use salt.
\newline\newline
\noindent \textbf{-salt}: use salt quando encriptando (este é o padrão).
\newline\newline
\noindent \textbf{-S <salt>}: o salt a ser usado: precisa ser representado como uma string de digitos hexadecimais.
\newline\newline

\noindent Exemplo de comando:
\begin{lstlisting}
# Codificar arquivo em base 64
$ openssl enc -base64 -in meu_texto.txt -out minha_saida_base64.txt
$ cat minha_saida_base64.txt
SSBhbSBnb2luIHRvIGJlIGVuY29kZWQK
\end{lstlisting}

\section*{4) Comando passwd}
O comando PASSWD da ferramenta OpenSSL tem como objetivo fornecer meios para encriptar ou descriptografar vários blocos ou cifradores stream baseado em senhas ou explicitamente fornecidos. Encriptação e descriptografia também podem ser feitas em Base64.
\newline\newline
\noindent \textbf{-help}: printa uma mensagem de como utilizar.
\newline\newline
\noindent \textbf{-crypt}: Usa o algoritmo de crypt.
\newline\newline
\noindent \textbf{-1}: use o MD5 baseado no algoritmo de senhas BSD 1.
\newline\newline
\noindent \textbf{-apr1}: usa o algoritmo apr1 (variante do algoritmo BSD da Apache.
\newline\newline
\noindent \textbf{-5}
\textbf{-6}: usa o algoritmo SHA256/SHA512 definidos por Ulrich Drepper. 
\newline\newline
\noindent \textbf{-salt <string>}: usa o salt especificado. Quando está lendo uma senha do terminal, implica diretamente em -noverify.
\newline\newline
\noindent \textbf{-in <arquivo>}: lê a senha do arquivo.
\newline\newline
\noindent \textbf{-stdin}: lê senhas da entrada padrão.
\newline\newline
\noindent \textbf{-noverify}: não verifica quando lendo uma senha do terminal.
\newline\newline
\noindent \textbf{-quiet}: não gera avisos de saída quando as senhas usadas na linha de comando estão truncadas.
\newline\newline
\noindent \textbf{-table}: na lista de saída, adiciona a senha em \textit{cleartext} e uma caracter de TAB para cada hash de senha.
\newline\newline

\noindent Exemplo de comando:
\begin{lstlisting}
$ openssl passwd -1 -salt
Password:
$1$yoursalt$5WA5NN0quMJ62v5LCu8kj1
\end{lstlisting}

\section*{5) Comando rand}
O comando RAND da ferramenta OpenSSL tem como objetivo fornecer números pseudo-aleatórios gerados a partir de um seed. 
\newline\newline
\noindent \textbf{-help}: printa na tela uma mensagem de como usar este comando.
\newline\newline
\noindent \textbf{-out <arquivo>}: escreve a saída para um arquivo ao invés da saída padrão.
\newline\newline
\noindent \textbf{-rand <arquivo(s)>}: usa arquivos especificados para usar de seed para o gerador de números pseudo-aleatório. Múltiplos arquivos podem ser especificados de acordo com o sistema operacional: ";" para Windows, "," para OpenVMS e ":" para todos os outros.
\newline\newline
\noindent \textbf{-base64}: faz uma codificação de base64 da saída.
\newline\newline
\noindent \textbf{-hex}: mostra a saída como uma string em hexadecimal.
\newline\newline

\noindent Exemplo de comando:
\begin{lstlisting}
# Gera 8 bits aleatorios e armazena no arquivo rand1.bin
$ openssl rand 8 -out rand1.bin
$ xxd rand1.bin
00000000: 52c4 c60a 9ab9 3279                      R.....2y
\end{lstlisting}

\section*{6) Comando speed}
O comando SPEED da ferramenta OpenSSL tem como objetivo testar a performance de algoritmos criptográficos. Bom para realizar \textit{benchmarks}.
\newline\newline
\noindent \textbf{-help}: printa uma mensagem com instruções de uso para o comando.
\newline\newline
\noindent \textbf{-engine <id>}: especificar um motor (de acordo com o seu ID único) fará com que o comando tente obter uma referência funcional para o motor especificado, portanto inicializando-o se for necessário. 
\newline\newline
\noindent \textbf{-elapsed}: mensura o tempo em tempo real ao invés de tempo da CPU. Pode ser útil para testar a velocidade em hardwares específicos.
\newline\newline
\noindent \textbf{-evp <algo>}: usa o cifrador especificado ou o algoritmo de message digest pela interface EVP.
\newline\newline
\noindent \textbf{-decrypt}: vai verificar o tempo de decriptação ao invés do tempo de encriptação. Affeta apenas o teste EVP. 
\newline\newline
\noindent \textbf{[zero or more test algorithms]}: se alguma opção é dada, então usa o teste de tempo para esses algoritmos. Do contrário, todos os algoritmos acima são utilizados.
\newline\newline

Exemplo de comando:
\begin{lstlisting}
# Faz um benchmark da velocidade dos algoritmos de encriptacao no meu computador.
$ openssl speed
Doing md4 for 3s on 16 size blocks: 13621592 md4's in 3.00s
Doing md4 for 3s on 64 size blocks: 10239804 md4's in 3.00s
Doing md4 for 3s on 256 size blocks: 5901982 md4's in 3.00s
Doing md4 for 3s on 1024 size blocks: 2186520 md4's in 3.00s
Doing md4 for 3s on 8192 size blocks: 319187 md4's in 3.00s
Doing md5 for 3s on 16 size blocks: 7905998 md5's in 3.00s
Doing md5 for 3s on 64 size blocks: 5532097 md5's in 3.00s
Doing md5 for 3s on 256 size blocks: 3151240 md5's in 3.00s
Doing md5 for 3s on 1024 size blocks: 1283545 md5's in 3.00s
Doing md5 for 3s on 8192 size blocks: 187701 md5's in 3.00s
Doing hmac(md5) for 3s on 16 size blocks: 8009845 hmac(md5)'s in 3.00s
Doing hmac(md5) for 3s on 64 size blocks: 6083095 hmac(md5)'s in 3.00s
Doing hmac(md5) for 3s on 256 size blocks: 3492758 hmac(md5)'s in 3.00s
Doing hmac(md5) for 3s on 1024 size blocks: 1281680 hmac(md5)'s in 3.00s
Doing hmac(md5) for 3s on 8192 size blocks: 186728 hmac(md5)'s in 2.99s
Doing sha1 for 3s on 16 size blocks: 11334892 sha1's in 3.00s
Doing sha1 for 3s on 64 size blocks: 7635323 sha1's in 3.00s
Doing sha1 for 3s on 256 size blocks: 4589719 sha1's in 3.00s
Doing sha1 for 3s on 1024 size blocks: 1699105 sha1's in 3.00s
Doing sha1 for 3s on 8192 size blocks: 245389 sha1's in 3.00s
Doing sha256 for 3s on 16 size blocks: 10155012 sha256's in 3.00s
Doing sha256 for 3s on 64 size blocks: 5905678 sha256's in 3.00s
Doing sha256 for 3s on 256 size blocks: 2952707 sha256's in 3.00s
Doing sha256 for 3s on 1024 size blocks: 869493 sha256's in 3.00s
Doing sha256 for 3s on 8192 size blocks: 109402 sha256's in 3.00s
Doing sha512 for 3s on 16 size blocks: 6731332 sha512's in 3.00s
Doing sha512 for 3s on 64 size blocks: 6870264 sha512's in 3.00s
Doing sha512 for 3s on 256 size blocks: 3305895 sha512's in 3.00s
Doing sha512 for 3s on 1024 size blocks: 1147902 sha512's in 3.00s
Doing sha512 for 3s on 8192 size blocks: 165633 sha512's in 3.00s
Doing whirlpool for 3s on 16 size blocks: 4335960 whirlpool's in 3.00s
Doing whirlpool for 3s on 64 size blocks: 2577480 whirlpool's in 3.00s
Doing whirlpool for 3s on 256 size blocks: 1069801 whirlpool's in 3.00s
Doing whirlpool for 3s on 1024 size blocks: 320088 whirlpool's in 3.00s
Doing whirlpool for 3s on 8192 size blocks: 41874 whirlpool's in 3.00s
Doing rmd160 for 3s on 16 size blocks: 6175293 rmd160's in 3.00s
Doing rmd160 for 3s on 64 size blocks: 3839459 rmd160's in 3.00s
Doing rmd160 for 3s on 256 size blocks: 1738074 rmd160's in 3.00s
Doing rmd160 for 3s on 1024 size blocks: 531218 rmd160's in 3.00s
Doing rmd160 for 3s on 8192 size blocks: 67192 rmd160's in 3.00s
Doing rc4 for 3s on 16 size blocks: 54556333 rc4's in 2.99s
Doing rc4 for 3s on 64 size blocks: 23391953 rc4's in 3.00s
Doing rc4 for 3s on 256 size blocks: 6939078 rc4's in 3.00s
Doing rc4 for 3s on 1024 size blocks: 1777970 rc4's in 3.00s
Doing rc4 for 3s on 8192 size blocks: 225977 rc4's in 3.00s
Doing des cbc for 3s on 16 size blocks: 10740581 des cbc's in 3.00s
Doing des cbc for 3s on 64 size blocks: 2758040 des cbc's in 3.00s
Doing des cbc for 3s on 256 size blocks: 680809 des cbc's in 3.00s
Doing des cbc for 3s on 1024 size blocks: 171407 des cbc's in 3.00s
Doing des cbc for 3s on 8192 size blocks: 21347 des cbc's in 3.00s
Doing des ede3 for 3s on 16 size blocks: 4067650 des ede3's in 3.00s
Doing des ede3 for 3s on 64 size blocks: 1021482 des ede3's in 3.00s
Doing des ede3 for 3s on 256 size blocks: 255168 des ede3's in 3.00s
Doing des ede3 for 3s on 1024 size blocks: 64499 des ede3's in 3.00s
Doing des ede3 for 3s on 8192 size blocks: 8071 des ede3's in 2.98s
Doing aes-128 cbc for 3s on 16 size blocks: 19556081 aes-128 cbc's in 3.00s
Doing aes-128 cbc for 3s on 64 size blocks: 5268901 aes-128 cbc's in 2.98s
Doing aes-128 cbc for 3s on 256 size blocks: 1382784 aes-128 cbc's in 3.00s
Doing aes-128 cbc for 3s on 1024 size blocks: 344760 aes-128 cbc's in 3.00s
Doing aes-128 cbc for 3s on 8192 size blocks: 42878 aes-128 cbc's in 3.00s
Doing aes-192 cbc for 3s on 16 size blocks: 15276462 aes-192 cbc's in 3.00s
Doing aes-192 cbc for 3s on 64 size blocks: 3987925 aes-192 cbc's in 3.00s
Doing aes-192 cbc for 3s on 256 size blocks: 1033564 aes-192 cbc's in 3.00s
Doing aes-192 cbc for 3s on 1024 size blocks: 265124 aes-192 cbc's in 3.00s
Doing aes-192 cbc for 3s on 8192 size blocks: 33892 aes-192 cbc's in 3.00s
Doing aes-256 cbc for 3s on 16 size blocks: 12975236 aes-256 cbc's in 3.00s
Doing aes-256 cbc for 3s on 64 size blocks: 3177106 aes-256 cbc's in 2.99s
Doing aes-256 cbc for 3s on 256 size blocks: 892504 aes-256 cbc's in 3.00s
Doing aes-256 cbc for 3s on 1024 size blocks: 221622 aes-256 cbc's in 3.00s
Doing aes-256 cbc for 3s on 8192 size blocks: 27547 aes-256 cbc's in 3.00s
Doing aes-128 ige for 3s on 16 size blocks: 18974989 aes-128 ige's in 2.99s
Doing aes-128 ige for 3s on 64 size blocks: 4584223 aes-128 ige's in 3.00s
Doing aes-128 ige for 3s on 256 size blocks: 1165075 aes-128 ige's in 3.00s
Doing aes-128 ige for 3s on 1024 size blocks: 304980 aes-128 ige's in 2.98s
Doing aes-128 ige for 3s on 8192 size blocks: 35839 aes-128 ige's in 3.00s
Doing aes-192 ige for 3s on 16 size blocks: 16553145 aes-192 ige's in 3.00s
Doing aes-192 ige for 3s on 64 size blocks: 4080884 aes-192 ige's in 3.00s
Doing aes-192 ige for 3s on 256 size blocks: 1046068 aes-192 ige's in 2.99s
Doing aes-192 ige for 3s on 1024 size blocks: 222882 aes-192 ige's in 3.00s
Doing aes-192 ige for 3s on 8192 size blocks: 28026 aes-192 ige's in 3.00s
Doing aes-256 ige for 3s on 16 size blocks: 13182810 aes-256 ige's in 3.00s
Doing aes-256 ige for 3s on 64 size blocks: 3061727 aes-256 ige's in 3.00s
Doing aes-256 ige for 3s on 256 size blocks: 793723 aes-256 ige's in 2.98s
Doing aes-256 ige for 3s on 1024 size blocks: 209393 aes-256 ige's in 2.99s
Doing aes-256 ige for 3s on 8192 size blocks: 26010 aes-256 ige's in 3.00s
Doing ghash for 3s on 16 size blocks: 161509112 ghash's in 3.00s
Doing ghash for 3s on 64 size blocks: 132800014 ghash's in 3.00s
Doing ghash for 3s on 256 size blocks: 56504797 ghash's in 3.00s
Doing ghash for 3s on 1024 size blocks: 15203044 ghash's in 2.97s
Doing ghash for 3s on 8192 size blocks: 2244381 ghash's in 2.99s
Doing camellia-128 cbc for 3s on 16 size blocks: 13378774 camellia-128 cbc's in 3.00s
Doing camellia-128 cbc for 3s on 64 size blocks: 5224896 camellia-128 cbc's in 3.00s
Doing camellia-128 cbc for 3s on 256 size blocks: 1638421 camellia-128 cbc's in 3.00s
Doing camellia-128 cbc for 3s on 1024 size blocks: 417163 camellia-128 cbc's in 3.00s
Doing camellia-128 cbc for 3s on 8192 size blocks: 50035 camellia-128 cbc's in 3.00s
Doing camellia-192 cbc for 3s on 16 size blocks: 11808534 camellia-192 cbc's in 3.00s
Doing camellia-192 cbc for 3s on 64 size blocks: 4182872 camellia-192 cbc's in 3.00s
Doing camellia-192 cbc for 3s on 256 size blocks: 1166638 camellia-192 cbc's in 3.00s
Doing camellia-192 cbc for 3s on 1024 size blocks: 294774 camellia-192 cbc's in 3.00s
Doing camellia-192 cbc for 3s on 8192 size blocks: 36936 camellia-192 cbc's in 3.00s
Doing camellia-256 cbc for 3s on 16 size blocks: 12261399 camellia-256 cbc's in 3.00s
Doing camellia-256 cbc for 3s on 64 size blocks: 4120629 camellia-256 cbc's in 3.00s
Doing camellia-256 cbc for 3s on 256 size blocks: 1133325 camellia-256 cbc's in 3.00s
Doing camellia-256 cbc for 3s on 1024 size blocks: 308700 camellia-256 cbc's in 3.00s
Doing camellia-256 cbc for 3s on 8192 size blocks: 37282 camellia-256 cbc's in 3.00s
Doing seed cbc for 3s on 16 size blocks: 11320701 seed cbc's in 3.00s
Doing seed cbc for 3s on 64 size blocks: 2741992 seed cbc's in 3.00s
Doing seed cbc for 3s on 256 size blocks: 714561 seed cbc's in 3.00s
Doing seed cbc for 3s on 1024 size blocks: 184665 seed cbc's in 3.00s
Doing seed cbc for 3s on 8192 size blocks: 23129 seed cbc's in 3.00s
Doing rc2 cbc for 3s on 16 size blocks: 7110354 rc2 cbc's in 3.00s
Doing rc2 cbc for 3s on 64 size blocks: 1885029 rc2 cbc's in 3.00s
Doing rc2 cbc for 3s on 256 size blocks: 475165 rc2 cbc's in 3.00s
Doing rc2 cbc for 3s on 1024 size blocks: 114167 rc2 cbc's in 3.00s
Doing rc2 cbc for 3s on 8192 size blocks: 14968 rc2 cbc's in 3.00s
Doing blowfish cbc for 3s on 16 size blocks: 17303000 blowfish cbc's in 3.00s
Doing blowfish cbc for 3s on 64 size blocks: 4516158 blowfish cbc's in 3.00s
Doing blowfish cbc for 3s on 256 size blocks: 1141510 blowfish cbc's in 2.99s
Doing blowfish cbc for 3s on 1024 size blocks: 287649 blowfish cbc's in 3.00s
Doing blowfish cbc for 3s on 8192 size blocks: 35200 blowfish cbc's in 3.00s
Doing cast cbc for 3s on 16 size blocks: 15842021 cast cbc's in 2.99s
Doing cast cbc for 3s on 64 size blocks: 4300098 cast cbc's in 3.00s
Doing cast cbc for 3s on 256 size blocks: 1095230 cast cbc's in 3.00s
Doing cast cbc for 3s on 1024 size blocks: 277468 cast cbc's in 3.00s
Doing cast cbc for 3s on 8192 size blocks: 34482 cast cbc's in 3.00s
Doing 512 bit private rsa's for 10s: 158504 512 bit private RSA's in 10.00s
Doing 512 bit public rsa's for 10s: 2037684 512 bit public RSA's in 9.99s
Doing 1024 bit private rsa's for 10s: 56185 1024 bit private RSA's in 9.99s
Doing 1024 bit public rsa's for 10s: 852498 1024 bit public RSA's in 10.00s
Doing 2048 bit private rsa's for 10s: 11842 2048 bit private RSA's in 10.00s
Doing 2048 bit public rsa's for 10s: 274765 2048 bit public RSA's in 10.00s
Doing 4096 bit private rsa's for 10s: 1177 4096 bit private RSA's in 10.00s
Doing 4096 bit public rsa's for 10s: 70746 4096 bit public RSA's in 10.00s
Doing 512 bit sign dsa's for 10s: 133438 512 bit DSA signs in 9.99s
Doing 512 bit verify dsa's for 10s: 161504 512 bit DSA verify in 9.99s
Doing 1024 bit sign dsa's for 10s: 64123 1024 bit DSA signs in 10.00s
Doing 1024 bit verify dsa's for 10s: 68694 1024 bit DSA verify in 10.00s
Doing 2048 bit sign dsa's for 10s: 22561 2048 bit DSA signs in 10.00s
Doing 2048 bit verify dsa's for 10s: 19803 2048 bit DSA verify in 10.00s
Doing 160 bit sign ecdsa's for 10s: 130122 160 bit ECDSA signs in 10.00s 
Doing 160 bit verify ecdsa's for 10s: 37711 160 bit ECDSA verify in 10.00s
Doing 192 bit sign ecdsa's for 10s: 116077 192 bit ECDSA signs in 10.00s 
Doing 192 bit verify ecdsa's for 10s: 30791 192 bit ECDSA verify in 9.99s
Doing 224 bit sign ecdsa's for 10s: 121442 224 bit ECDSA signs in 10.00s 
Doing 224 bit verify ecdsa's for 10s: 55749 224 bit ECDSA verify in 10.00s
Doing 256 bit sign ecdsa's for 10s: 180561 256 bit ECDSA signs in 10.00s 
Doing 256 bit verify ecdsa's for 10s: 63484 256 bit ECDSA verify in 10.00s
Doing 384 bit sign ecdsa's for 10s: 37804 384 bit ECDSA signs in 10.00s 
Doing 384 bit verify ecdsa's for 10s: 8663 384 bit ECDSA verify in 10.00s
Doing 521 bit sign ecdsa's for 10s: 20125 521 bit ECDSA signs in 10.00s 
Doing 521 bit verify ecdsa's for 10s: 10814 521 bit ECDSA verify in 9.99s
Doing 163 bit sign ecdsa's for 10s: 42827 163 bit ECDSA signs in 9.96s 
Doing 163 bit verify ecdsa's for 10s: 18169 163 bit ECDSA verify in 9.99s
Doing 233 bit sign ecdsa's for 10s: 22586 233 bit ECDSA signs in 9.99s 
Doing 233 bit verify ecdsa's for 10s: 15181 233 bit ECDSA verify in 9.95s
Doing 283 bit sign ecdsa's for 10s: 15516 283 bit ECDSA signs in 9.97s 
Doing 283 bit verify ecdsa's for 10s: 8420 283 bit ECDSA verify in 9.96s
Doing 409 bit sign ecdsa's for 10s: 6313 409 bit ECDSA signs in 10.00s 
Doing 409 bit verify ecdsa's for 10s: 4646 409 bit ECDSA verify in 10.00s
Doing 571 bit sign ecdsa's for 10s: 2722 571 bit ECDSA signs in 10.00s 
Doing 571 bit verify ecdsa's for 10s: 1966 571 bit ECDSA verify in 9.99s
Doing 163 bit sign ecdsa's for 10s: 40800 163 bit ECDSA signs in 9.98s 
Doing 163 bit verify ecdsa's for 10s: 17618 163 bit ECDSA verify in 10.00s
Doing 233 bit sign ecdsa's for 10s: 23682 233 bit ECDSA signs in 10.00s 
Doing 233 bit verify ecdsa's for 10s: 14473 233 bit ECDSA verify in 10.00s
Doing 283 bit sign ecdsa's for 10s: 15560 283 bit ECDSA signs in 10.00s 
Doing 283 bit verify ecdsa's for 10s: 7915 283 bit ECDSA verify in 10.00s
Doing 409 bit sign ecdsa's for 10s: 6716 409 bit ECDSA signs in 9.95s 
Doing 409 bit verify ecdsa's for 10s: 4906 409 bit ECDSA verify in 9.98s
Doing 571 bit sign ecdsa's for 10s: 2919 571 bit ECDSA signs in 9.99s 
Doing 571 bit verify ecdsa's for 10s: 1829 571 bit ECDSA verify in 10.00s
Doing 160 bit  ecdh's for 10s: 47214 160-bit ECDH ops in 10.00s
Doing 192 bit  ecdh's for 10s: 39533 192-bit ECDH ops in 10.00s
Doing 224 bit  ecdh's for 10s: 82735 224-bit ECDH ops in 9.99s
Doing 256 bit  ecdh's for 10s: 97678 256-bit ECDH ops in 9.99s
Doing 384 bit  ecdh's for 10s: 12379 384-bit ECDH ops in 10.00s
..................................
..................................
\end{lstlisting}

\end{document}
